\documentclass[10pt]{jsarticle}
\pagestyle{empty}
\usepackage[dvipdfmx]{graphicx}
\usepackage[labelformat=simple]{subcaption}
\AtBeginDvi{\special{papersize=210mm,297mm}}
\usepackage{amsmath}
\usepackage{amsfonts}
\usepackage{bm}
%\usepackage[psamsfonts]{amssymb}
%\usepackage{txfonts}
%\usepackage{type1cm}
%\usepackage{latexsym}
%\usepackage{psfrag}
\usepackage{enumitem}
\renewcommand\thesubfigure{(\alph{subfigure})}

\begin{document}

\begin{center}
  {\bf \Large 信号処理のための数学365本ノック}
\end{center}
\begin{flushright}
  Yuma Kinoshita
\end{flushright}
\setlength{\leftmarginii}{3pt}

\begin{enumerate}[label=問\arabic*.]
  \vspace{2mm}
  % 問1
  \item てすと
  \vspace{1mm}
  \begin{enumerate}[label=(\arabic*)]
    \item まず,標準内積
      $<\cdot, \cdot>_{\mathbb{C}^n}: \mathbb{C}^n \times \mathbb{C}^n \rightarrow \mathbb{R}$を持つ
      $n$次元計量ベクトル空間$(\mathbb{C}^n, <\cdot, \cdot>_{\mathbb{C}^n})$の場合について証明する.

      エルミート行列$\mathbf{A}$の固有値$\lambda_i, \lambda_j (i \neq j, \lambda_i \neq \lambda_j)$
      とそれぞれに対応する固有ベクトル$\bm{u}_i, \bm{u}_j$を考える.
      \begin{align}
        \bm{u}_i^* \mathbf{A} \bm{u}_i &= \lambda_i \bm{u}_i^* \bm{u}_i, \\
        (\mathbf{A} \bm{u}_i)^* \bm{u}_i &= (\lambda_i \bm{u}_i)^* \bm{u}_i
          = \overline{\lambda_i} \bm{u}_i^* \bm{u}_i
      \end{align}
      および
      \begin{align}
        \bm{u}_i^* \mathbf{A} \bm{u}_i = \bm{u}_i^* \mathbf{A}^* \bm{u}_i
          = (\mathbf{A} \bm{u}_i)^* \bm{u}_i
      \end{align}
      より,$\lambda_i = \overline{\lambda}_i$.
      したがって,エルミート行列$\mathbf{A}$の固有値は実数である.

      また,$\bm{x}, \bm{y} \in \mathbb{C}^n$について
      $<\bm{x}, \bm{y}>_{\mathbb{C}^n} = \bm{x}^* \bm{y}$であるから,
      \begin{align}
        <\bm{u}_i, \mathbf{A} \bm{u}_j>_{\mathbb{C}^n} &= \bm{u}_i^* \mathbf{A} \bm{u}_j
          = \lambda_j \bm{u}_i^* \bm{u}_j, \\
        <\mathbf{A} \bm{u}_i, \bm{u}_j>_{\mathbb{C}^n} &= \overline{\lambda_i} \bm{u}_i^* \bm{u}_j
          = \lambda_i \bm{u}_i^* \bm{u}_j
      \end{align}
      となる.
      ここで,
      \begin{align}
        & <\bm{u}_i, \mathbf{A} \bm{u}_j>_{\mathbb{C}^n} \\
          &= \bm{u}_i^* \mathbf{A} \bm{u}_j = \bm{u}_i^* \mathbf{A}^* \bm{u}_j
          = (\mathbf{A} \bm{u}_i)^* \bm{u}_j, \\
          &= <\mathbf{A} \bm{u}_i, \bm{u}_j>_{\mathbb{C}^n}
          \label{eq:1}
      \end{align}
      より,$\lambda_j \bm{u}_i^* \bm{u}_j = \lambda_i \bm{u}_i^* \bm{u}_j$.
      $\lambda_i \neq \lambda_j$だから
      $\bm{u}_i^* \bm{u}_j = <\bm{u}_i, \bm{u}_j>_{\mathbb{C}^n} = 0$.
      したがって,固有ベクトル$\bm{u}_i, \bm{u}_j$は標準内積のもとで直交する.
      
      次に,任意の内積
      $<\cdot, \cdot>: \mathbb{C}^n \times \mathbb{C}^n \rightarrow \mathbb{R}$を持つ
      $n$次元計量ベクトル空間$V = (\mathbb{C}^n, <\cdot, \cdot>)$の場合について証明する.

      $V$の正規直交基底$\{\bm{e}_1, \cdots, \bm{e}_n\}$に関する$\bm{a}, \bm{b}$の座標を
      それぞれ$\bm{x}=(x_1, \cdots, x_n)^\top, \bm{y}=(y_1, \cdots, y_n)^\top$とすれば,
      \begin{align}
        <\bm{a}, \bm{b}> \\
          &= <x_1 \bm{e}_1 + \cdots + x_n \bm{e}_n, y_1 \bm{e}_1 + \cdots + y_n \bm{e}_m> \\
          &= \sum_{i, j = 1, \cdots, n} <x_i \bm{e}_i, y_j \bm{e}_j>
          = <\bm{x}, \bm{y}>_{\mathbb{C}^n}
          \label{eq:2}
      \end{align}
      が成り立つ.
      式(\ref{eq:1})および式(\ref{eq:2})から,エルミート行列$\mathbf{A}$について
      \begin{equation}
        <\mathbf{A} \bm{a}, \bm{b}> = <\mathbf{A} \bm{x}, \bm{y}>_{\mathbb{C}^n}
          = <\bm{x}, \mathbf{A} \bm{y}>_{\mathbb{C}^n} = <\bm{a}, \mathbf{A} \bm{b}>
      \end{equation}
      が成立する.
      また,内積は$\alpha, \beta \in \mathbb{C}$について以下の性質を満たす.
      \begin{align}
        <\alpha \bm{a}, \bm{b}> &= \overline{\alpha} <\bm{a}, \bm{b}> \\
        <\bm{a}, \beta \bm{b}> &= \beta <\bm{a}, \bm{b}>
      \end{align}
      したがって,$\mathbf{A}$の固有ベクトル$\bm{u}_i, \bm{u}_j$について
      \begin{align}
        <\bm{u}_i, \mathbf{A} \bm{u}_j> &= <\bm{u}_i, \lambda_j \bm{u}_j> = \lambda_j <\bm{u}_i, \bm{u}_j>, \\
        <\mathbf{A} \bm{u}_i, \bm{u}_j> &= <\lambda_i \bm{u}_i, \bm{u}_j> = \lambda_i <\bm{u}_i, \bm{u}_j>, \\
        <\bm{u}_i, \mathbf{A} \bm{u}_j> &= <\mathbf{A} \bm{u}_i, \bm{u}_j>
      \end{align}
      が成り立つから,
      $\lambda_i \neq \lambda_j$より
      $<\bm{u}_i, \bm{u}_j> = 0$.
      したがって,固有ベクトル$\bm{u}_i, \bm{u}_j$は任意の内積について直交する.

    \item 固有値分解$\mathbf{A} = \mathbf{U}\mathbf{\Lambda}\mathbf{U}^{-1}$における行列$\mathbf{U}$は,
      $\mathbf{A}$の固有ベクトルを並べた行列である.
      1の結果から,$\mathbf{A}$の固有ベクトルは直交するため,長さが1となるように固有ベクトルを選べば
      $\mathbf{U}$は正規直交基底を並べたベクトルとなる.
      したがって$\mathbf{U}$はユニタリ行列であり,
      $\mathbf{A} = \mathbf{U}\mathbf{\Lambda}\mathbf{U}^*$として与えられる.
    \item $\mathbf{X} = \mathbf{U}\mathbf{\Lambda}^{-1}\mathbf{U}^*$とすると
      \begin{equation}
        \mathbf{X}\mathbf{A}
          = \mathbf{U}\mathbf{\Lambda}^{-1}\mathbf{U}^* \mathbf{U}\mathbf{\Lambda}\mathbf{U}^*
          = \mathbf{E}.
      \end{equation}
      同様に,$\mathbf{A}\mathbf{X} = \mathbf{E}$.
      以上より,$\mathbf{X} = \mathbf{A}^{-1}$.
  \end{enumerate}
  
\end{enumerate}
\end{document}
