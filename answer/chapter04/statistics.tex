\section{確率統計の基礎}

\subsection{最小二乗法の閉形式解}
  \begin{enumerate}[label=(\roman*)]
    \item
      \begin{align}
        \mathcal{L}(\vecsym{x}) &= \tp{(\matsym{A}\vecsym{x} - \vecsym{b})} (\matsym{A}\vecsym{x} - \vecsym{b}) \\
          &= (\tp{\vecsym{x}}\tp{\matsym{A}} - \tp{\vecsym{b}}) (\matsym{A}\vecsym{x} - \vecsym{b}) \\
          &= \tp{\vecsym{x}}\tp{\matsym{A}}\matsym{A}\vecsym{x} - \tp{\vecsym{x}}\tp{\matsym{A}}\vecsym{b} - \tp{\vecsym{b}}\matsym{A}\vecsym{x} - \tp{\vecsym{b}}\vecsym{b} \\
          &= \tp{\vecsym{x}}\tp{\matsym{A}}\matsym{A}\vecsym{x} - 2\tp{\vecsym{x}}\tp{\matsym{A}}\vecsym{b} - \tp{\vecsym{b}}\vecsym{b}
      \end{align}
      より,
      \begin{equation}
        \frac{\partial}{\partial \vecsym{x}} \mathcal{L}(\vecsym{x}) = 2 \tp{\matsym{A}}\matsym{A}\vecsym{x} - 2\tp{\matsym{A}}\vecsym{b}
      \end{equation}
    \item $\mathcal{L}(\vecsym{x})$を最小にする$\vecsym{x}$は,
      \begin{equation}
        \frac{\partial}{\partial \vecsym{x}} \mathcal{L}(\vecsym{x}) = \vecsym{0}
      \end{equation}
      を満たす.
      したがって,$\vecsym{x}$が満たすべき条件は,
      \begin{equation}
        \tp{\matsym{A}}\matsym{A}\vecsym{x} = \tp{\matsym{A}}\vecsym{b}
      \end{equation}
      (正規方程式)
    \item $\tp{\matsym{A}}\matsym{A}$が正則であることから,
      \begin{equation}
        \vecsym{x} = (\tp{\matsym{A}}\matsym{A})^{-1}\tp{\matsym{A}}\vecsym{b}
      \end{equation}
  \end{enumerate}