\section{線形代数の基礎}

\subsection{連立方程式\label{linalg:linear_system}}
  \vspace{1mm}
  \begin{enumerate}[label=(\roman*)]
    \item
    拡大係数行列$\tilde{\mathrm{A}} = (\mathrm{A}|b)$に対し,拡大係数行列の左側が以下のような行列
    (対角に相当する成分のみ1を含み,他の成分がすべて0である行列) となるよう,
    行基本変形を施す.
    \begin{equation}
      \left(
        \begin{array}{rrrrrrr}
          1 &   &        &   &   &        & 0 \\
            & 1 &        &   &   &        &   \\
            &   & \ddots &   &   &        &   \\
            &   &        & 1 &   &        &   \\
            &   &        &   & 0 &        &   \\
            &   &        &   &   & \ddots &   \\
          0 &   &        &   &   &        & 0 \\
        \end{array}
      \right)
    \end{equation}
    この結果得られる拡大係数行列の右側が元の方程式の解に相当する.
    よって解$x$は,
    \begin{equation}
      x = \left(
        \begin{array}{r}
          6 \\ 2 \\ 3
        \end{array}
        \right)
    \end{equation}
    である.
    
    \item
    \begin{itemize}
      \item (i)と同様に拡大係数行列$\tilde{\mathrm{A}} = (\mathrm{A}|b)$を変形すると以下を得る
        (変形を途中でやめていることに注意).
        \begin{equation}
          \left(
            \begin{array}{rr|r}
              1 & 1 & 1 \\
              0 & 1 & 0 \\
              0 & 0 & 2 \\
            \end{array}
          \right)
        \end{equation}
        このことから,
        $\rank(\mathrm{A}) \neq \rank(\tilde{\mathrm{A}})$であり,
        $Ax - b = 0$は解を持たない.
      \item $\|\mathrm{A}x-b\|$ を最小とする $x$ は,
      $\|\mathrm{A}x-b\|^2$ を最小とする $x$ と等しい.
      よって,$\|\mathrm{A}x-b\|^2$ を最小とする $x$ を考える.
      $\|\mathrm{A}x-b\|^2$は二次式だから,
      \begin{equation}
        \frac{\partial}{\partial x} \| \mathrm{A}x-b \|^2 = 0
      \end{equation}
      を満たす$x$を求めればよい.
      上式を解くと,
      \begin{equation}
        x = (\mathrm{A}^\top \mathrm{A})^{-1} \mathrm{A}^\top b
      \end{equation}
      を得る.
      したがって,
      \begin{equation}
        x = \left(
          \begin{array}{r}
             1 \\
             \frac{2}{3} \\
          \end{array}
          \right).
      \end{equation}
      \begin{enumerate}[label=別解\arabic*]
        \item $\|\mathrm{A}x-b\|^2$を$x$の各要素について平方完成して最小値を求める
        \item $\|\mathrm{A}x-b\|^2$を$x$の各要素について微分して,
          \begin{equation}
            \frac{\partial}{\partial x_1} \| \mathrm{A}x-b \|^2
            = \frac{\partial}{\partial x_2} \| \mathrm{A}x-b \|^2 = 0
          \end{equation}
          を解き,最小値を求める.
      \end{enumerate}
    \end{itemize}

    \item
    \begin{itemize}
      \item (i)と同様に拡大係数行列$\tilde{\mathrm{A}} = (\mathrm{A}|b)$を変形すると以下を得る.
        \begin{equation}
          \left(
            \begin{array}{rrrr|r}
              1 & 0 & 0 &  2 & -1 \\
              0 & 1 & 0 & -1 &  1 \\
              0 & 0 & 1 & -1 & -1 \\
              0 & 0 & 0 &  0 &  0 \\
            \end{array}
          \right)
        \end{equation}
        よって,解$x$は,任意の$t \in \mathbb{R}$を用いて
        \begin{equation}
          x = t
            \left(
            \begin{array}{r}
              -2 \\ 1 \\ 1 \\ 1
            \end{array}
            \right)
            +
            \left(
            \begin{array}{r}
              -1 \\ 1 \\ -1 \\ 0
            \end{array}
            \right)
        \end{equation}
        として与えられる.
      \item $\|x\|$ を最小とする $x$ は,
      $\|x\|^2$ を最小とする $x$ と等しい.
      よって,$\|x\|^2$ を最小とする $x$ を考える.
      $\|x\|^2$は二次式だから,
      \begin{equation}
        \frac{d}{dt} \| x \|^2 = 0
      \end{equation}
      を満たす$x$を求めればよい.
      上式を解くと,
      \begin{equation}
        t = -\frac{2}{7}
      \end{equation}
      を得る.
      したがって,
      \begin{equation}
        x = -\frac{1}{7}
          \left(
          \begin{array}{r}
              3 \\
             -5 \\
              9 \\
              2
          \end{array}
          \right).
      \end{equation}
      \begin{enumerate}[label=別解\arabic*]
        \item $\|\mathrm{A}x-b\|^2$を$t$について平方完成して最小値を求める
      \end{enumerate}
    \end{itemize}
  \end{enumerate}

\clearpage
\subsection{線形変換}
  \vspace{1mm}
  \begin{enumerate}[label=(\roman*)]
    \item $\vecsym{y} = a \vecsym{x} + \vecsym{x}_\perp$より,
      $\vecsym{x}^\top \vecsym{y} = \vecsym{x}^\top (a \vecsym{x} + \vecsym{x}_\perp) = a \| \vecsym{x} \|^2$.
      $a$について整理すると,
      \begin{equation}
        a = \frac{1}{\| \vecsym{x} \|^2} \vecsym{x}^\top \vecsym{y}
      \end{equation}
      を得る.
    \item $\matsym{A} \vecsym{y} = a \vecsym{x}$に(i)の結果を代入することで,
      \begin{equation}
        \matsym{A} = \frac{1}{\| \vecsym{x} \|^2} \vecsym{x} \vecsym{x}^\top
      \end{equation}
      を得る.
  \end{enumerate}

\clearpage
\subsection{行列のランク}
  \vspace{1mm}
  \begin{enumerate}[label=(\roman*)]
    \item a
    \item b
    \item c
    \item d
    \item e
  \end{enumerate}

\clearpage
\subsection{行列式と逆行列の存在条件}
  \vspace{1mm}
  \begin{enumerate}[label=(\roman*)]
    \item
      \begin{align}
        \det{(\matsym{V})} &=
          \begin{vmatrix}
            1 & 1 & 1 & 1 \\
            x_1 & x_2 & x_3 & x_4 \\
            x_1^2 & x_2^2 & x_3^2 & x_4^2 \\
            x_1^3 & x_2^3 & x_3^3 & x_4^3
          \end{vmatrix}
          =
          \begin{vmatrix}
            1 & 1 & 1 & 1 \\
            0 & x_2 - x_1 & x_3 - x_1 & x_4 - x_1 \\
            0 & x_2 (x_2 - x_1) & x_3 (x_3 - x_1) & x_4 (x_4 - x_1)\\
            0 & x_2^2 (x_2 - x_1) & x_3^2 (x_3 - x_1) & x_4^2 (x_4 - x_1) 
          \end{vmatrix}\\
          &=
          \begin{vmatrix}
            x_2 - x_1 & x_3 - x_1 & x_4 - x_1 \\
            x_2 (x_2 - x_1) & x_3 (x_3 - x_1) & x_4 (x_4 - x_1)\\
            x_2^2 (x_2 - x_1) & x_3^2 (x_3 - x_1) & x_4^2 (x_4 - x_1) 
          \end{vmatrix}\\
          &= (x_2 - x_1) (x_3 - x_1) (x_4 - x_1)
          \begin{vmatrix}
            1 & 1 & 1 \\
            x_2 & x_3 & x_4\\
            x_2^2 & x_3^2 & x_4^2
          \end{vmatrix}\\
          &= (x_2 - x_1) (x_3 - x_1) (x_4 - x_1)
          \begin{vmatrix}
            1 & 1 & 1 \\
            0 & x_3 - x_2 & x_4 - x_2\\
            0 & x_3 (x_3 - x_2) & x_4 (x_3 - x_2)
          \end{vmatrix}\\
          &= (x_2 - x_1) (x_3 - x_1) (x_4 - x_1) (x_3 - x_2) (x_4 - x_2)
          \begin{vmatrix}
            1 & 1 \\
            x_3 & x_4
          \end{vmatrix}\\
          &= (x_2 - x_1) (x_3 - x_1) (x_4 - x_1) (x_3 - x_2) (x_4 - x_2) (x_4 - x_3)
      \end{align}
    \item 行列$\matsym{V}$が逆行列を持つための必要十分条件は,$\det{(\matsym{V})} \neq 0$であることである.
      \begin{equation}
        \det{(\matsym{V})} = (x_2 - x_1) (x_3 - x_1) (x_4 - x_1) (x_3 - x_2) (x_4 - x_2) (x_4 - x_3) \neq 0
      \end{equation}
      より,
      $i, j \in \{1, 2, 3, 4\}$かつ$i \neq j$であるすべての$(i, j)$について$x_i \neq x_j$を満たすとき,
      行列$\matsym{V}$は逆行列を持つ.
  \end{enumerate}
      
\clearpage
\subsection{回転行列}
  $\vecsym{e}_1 = \begin{pmatrix} 1 \\ 0 \end{pmatrix}$,
  $\vecsym{e}_2 = \begin{pmatrix} 0 \\ 1 \end{pmatrix}$
  とする.
  \begin{enumerate}[label=(\roman*)]
    \item $\vecsym{e}'_1 = \begin{pmatrix} \cos\theta \\ \sin\theta \end{pmatrix}$,
      $\vecsym{e}'_2 = \begin{pmatrix} -\sin\theta \\ \cos\theta \end{pmatrix}$
    \item 
      \begin{equation}
        \begin{pmatrix} \vecsym{e}'_1 & \vecsym{e}'_2 \end{pmatrix}
        = \matsym{R}_{\theta} \begin{pmatrix} \vecsym{e}_1 & \vecsym{e}_2 \end{pmatrix}
      \end{equation}
      より,
      \begin{equation}
        \begin{pmatrix}
          \cos\theta & -\sin\theta \\
          \sin\theta & \cos\theta
        \end{pmatrix}
        = \matsym{R}_{\theta}
        \begin{pmatrix}
          1 & 0 \\
          0 & 1
        \end{pmatrix}.
      \end{equation}
      したがって,
      \begin{equation}
        \matsym{R}_\theta =
          \begin{pmatrix}
            \cos\theta & -\sin\theta \\
            \sin\theta & \cos\theta
          \end{pmatrix}
      \end{equation}
    \item
      \begin{align}
        \vecsym{\nu} &= \matsym{R}_\phi \matsym{R}_\theta \vecsym{e}_1 \\
          &=
          \begin{pmatrix}
            \cos\theta & -\sin\theta \\
            \sin\theta & \cos\theta
          \end{pmatrix}
          \begin{pmatrix}
            \cos\theta & -\sin\theta \\
            \sin\theta & \cos\theta
          \end{pmatrix}
          \vecsym{e}_1\\
          &=
          \begin{pmatrix}
            \cos\theta \cos\phi - \sin\theta \sin\phi \\
            \sin\theta \cos\phi + \cos\theta \sin\phi
          \end{pmatrix}
      \end{align}
    \item
      \begin{equation}
        \matsym{R}_{-\theta} 
          =
          \begin{pmatrix}
            \cos\theta  & \sin\theta \\
            -\sin\theta & \cos\theta
          \end{pmatrix}
      \end{equation}
      であり,
      \begin{equation}
        \matsym{R}_\theta \matsym{R}_{-\theta} = \matsym{R}_{\theta} \matsym{R}_\theta = \matsym{I}
      \end{equation}
      ($\matsym{I}$は単位行列)
      より,$\matsym{R}_\theta^{-1} = \matsym{R}_{-\theta}$.
  \end{enumerate}


\clearpage
\subsection{エルミート行列}
  \begin{enumerate}[label=(\roman*)]
    \item 任意の行列$\matsym{M} \in \numset{C}^{n \times m}$について,
      $\matsym{M}\htp{\matsym{M}}$および
      $\htp{\matsym{M}}\matsym{M}$
      がエルミート行列となることを証明する.
      $\htp{(\matsym{M}\tilde{\matsym{M}})} = \htp{\tilde{\matsym{M}}}\htp{\matsym{M}}$より,
      $\htp{(\matsym{M}\htp{\matsym{M}})} = \htp{(\htp{\matsym{M}})}\htp{\matsym{M}} = \matsym{M}\htp{\matsym{M}}$
      であるため,$\matsym{M}\htp{\matsym{M}}$はエルミート行列.
      同様に,$\htp{\matsym{M}}\matsym{M}$もエルミート行列.
      
    \item 任意のエルミート行列$\matsym{A} \in \numset{C}^{n \times n}$について,
      すべての固有値が実数となることを証明する.
      定義より,行列$\matsym{A}$の固有値$\lambda$と固有ベクトル$\vecsym{v}$は,
      $\matsym{A}\vecsym{v} = \lambda \vecsym{v}$を満たす.
      よって,
      \begin{align}
        \htp{vecsym{v}} \matsym{A} \vecsym{v}
        = \htp{vecsym{v}} \lambda \vecsym{v}
        = \lambda \htp{vecsym{v}} \vecsym{v}.
      \end{align}
      また,
      \begin{align}
        \htp{vecsym{v}} \htp{\matsym{A}} \vecsym{v}
        = \htp{(\matsym{A} vecsym{v})} \vecsym{v}
        = \htp{(\lambda vecsym{v})} \vecsym{v}
        = \overline{\lambda} \htp{vecsym{v}} \vecsym{v}.
      \end{align}
      したがって,$\lambda = \overline{\lambda}$でなければならないため,
      $\lambda$は実数.
    
    \item 任意のエルミート行列$\matsym{A} \in \numset{C}^{n \times n}$について,
      $\matsym{A}$があるユニタリ行列$\matsym{U}$で対角化されることを証明する.
      
  \end{enumerate}

  \clearpage
  \subsection{特異値分解}
    \begin{enumerate}[label=(\roman*)]
      \item $\matsym{A} = \matsym{U}\matsym{\Sigma}\htp{\matsym{V}}$より,
        \begin{align}
          \matsym{A}\htp{\matsym{A}} &= \matsym{U}\matsym{\Sigma}\htp{\matsym{V}}\htp{(\matsym{U}\matsym{\Sigma}\htp{\matsym{V}})} \\
            &= \matsym{U}\matsym{\Sigma}\htp{\matsym{V}}\matsym{V}\htp{\matsym{\Sigma}}\htp{\matsym{U}}.
        \end{align}
        ここで,$\matsym{V}$はユニタリ行列だから,
        $\htp{\matsym{V}}\matsym{V} = \matsym{I}$ ($\matsym{I}$は単位行列).
        よって,
        \begin{equation}
          \matsym{A}\htp{\matsym{A}} = \matsym{U}\matsym{\Sigma}\htp{\matsym{\Sigma}}\htp{\matsym{U}}.
        \end{equation}
        $\matsym{\Sigma}\htp{\matsym{\Sigma}}$は対角行列となるため,
        $\matsym{\Sigma}\htp{\matsym{\Sigma}} = \matsym{\Lambda}$とおけば,
        \begin{equation}
          \matsym{A}\htp{\matsym{A}} = \matsym{U}\matsym{\Lambda}\htp{\matsym{U}}
        \end{equation}
        であり,これは,$\matsym{A}\htp{\matsym{A}}$の固有値分解である.
        よって$\matsym{U}$は,$\matsym{A}\htp{\matsym{A}}$の
        (ノルムが1に正規化された)固有ベクトルを並べた行列として与えられる.
      \item 同様に,
        $\htp{\matsym{\Sigma}}\matsym{\Sigma} = \matsym{\Lambda}'$とおけば,
        \begin{equation}
          \htp{\matsym{A}}\matsym{A} = \matsym{V}\matsym{\Lambda}'\htp{\matsym{V}}
        \end{equation}
        であり,これは,$\htp{\matsym{A}}\matsym{A}$の固有値分解である.
        よって$\matsym{V}$は,$\htp{\matsym{A}}\matsym{A}$の
        (ノルムが1に正規化された)固有ベクトルを並べた行列として与えられる.
      \item $\matsym{\Sigma}\htp{\matsym{\Sigma}} = \matsym{\Lambda}$であり,
        $\matsym{\Lambda}$の$(i, i)$要素は$\matsym{\Sigma}$の$(i, i)$要素の自乗である.
        また,
        $\matsym{\Lambda}$は$\matsym{A}\htp{\matsym{A}}$の固有値を対角に並べた行列である.
        したがって,$\matsym{\Sigma}$は,
        $\matsym{A}\htp{\matsym{A}}$の固有値の平方根を対角並べた行列となる.
      \item $\matsym{A}$が縦長行列であるため,
        $\matsym{A}\htp{\matsym{A}}$の固有値分解より
        $\htp{\matsym{A}}\matsym{A}$の固有値分解のほうが簡単である.
        $\htp{\matsym{A}}\matsym{A}$の固有値分解のほうが簡単である.
        \begin{align}
          \htp{\matsym{A}}\matsym{A}
            &= 
              \begin{pmatrix}
                1 & 1 & 0 \\
                1 & -1 & 1
              \end{pmatrix}
              \begin{pmatrix}
                1 & 1 \\
                1 & -1 \\
                0 & 1
              \end{pmatrix} \\
            &=
              \begin{pmatrix}
                2 & 0 \\
                0 & 3 
              \end{pmatrix}
        \end{align}
        より,固有値は$2, 3$であり,
        対応する固有ベクトルはそれぞれ$\tp{(a, 0)}, \tp{(0, b)}$である
        (ただし,$a, b$は任意のスカラー).
        よって,$a = b = 1$とすれば,
        \begin{equation}
          \matsym{V} = 
          \begin{pmatrix}
            1 & 0 \\
            0 & 1 
          \end{pmatrix},
          \matsym{\Sigma} =
          \begin{pmatrix}
            \sqrt{2} & 0 \\
            0 & \sqrt{3} \\
            0 & 0
          \end{pmatrix}
        \end{equation}
        を得る.
        また,$\matsym{A} = \matsym{U}\matsym{\Sigma}\htp{\matsym{V}}$より,
        $\matsym{U} = \matsym{A}\htp{\matsym{V}}$

    \end{enumerate}

  \clearpage
  \subsection{Moore-Penroseの擬似逆行列}
    方程式$\matsym{A}\vecsym{x} = \vecsym{b}$は,
    $\matsym{A}$が正則行列であれば,その逆行列を用いて
    $\vecsym{x} = \matsym{A}^{-1}\vecsym{b}$として解くことができる.
    任意の行列$\matsym{A}$については,
    その擬似逆行列を用いて
    $\vecsym{x} = \pinv{\matsym{A}}\vecsym{b}$として一つの(近似)解を得ることができる.
    その解は,
    \begin{itemize}[nosep]
      \item 方程式$\matsym{A}\vecsym{x} = \vecsym{b}$が決定系のときには,
        その解$\vecsym{x}$
      \item 方程式$\matsym{A}\vecsym{x} = \vecsym{b}$が優決定系のときには,
        $\|\matsym{A} \vecsym{x} - \vecsym{b} \|$ が最小となる $\vecsym{x}$
      \item 方程式$\matsym{A}\vecsym{x} = \vecsym{b}$が劣決定系のときには,
        $\matsym{A} \vecsym{x} - \vecsym{b} = \vecsym{0}$の解$\vecsym{x}$のうち,$\|\vecsym{x}\|$が最小となる$\vecsym{x}$
    \end{itemize}
    に一致する.
    
    \begin{enumerate}[label=(\roman*)]
      \item
        \begin{align}
          \pinv{\matsym{A}} &=
            \begin{pmatrix}
              \dfrac{1}{2} & \dfrac{1}{2} & 0 \\[1.5ex]
              \dfrac{1}{3} & -\dfrac{1}{3} & \dfrac{1}{3}
            \end{pmatrix} \\
          \pinv{\matsym{A}} \vecsym{b} &=
            \begin{pmatrix}
              1 \\[1.5ex]
              \dfrac{2}{3}
            \end{pmatrix} \\
        \end{align}
        よって,
        \ref{linalg:linear_system}(ii)で計算した$\vecsym{x}$と等しい.
      
      \item
        \begin{align}
          \pinv{\matsym{A}} &=
            \frac{1}{735}
            \begin{pmatrix}
               23 &  -26 & 143 &   9 \\
               97 &  146 & -68 &   6 \\
              -43 & -239 & 212 & 111 \\
              -8  &   41 & 142 & -99
            \end{pmatrix}\\
          \pinv{\matsym{A}} \vecsym{b} &=
            \frac{1}{7}
            \begin{pmatrix}
              -3 \\
               5 \\
              -9 \\
              -2 \\
            \end{pmatrix} \\
        \end{align}
        よって,
        \ref{linalg:linear_system}(iii)で計算した$\vecsym{x}$と等しい.
    \end{enumerate}
        

  \clearpage
  \subsection{深層学習?}
    $f_n(\vecsym{x}) = \vecsym{x}$より,
    $f(\vecsym{x})$は
    \begin{equation}
      f(\vecsym{x}) = \matsym{W}_k (\matsym{W}_{k-1} (\cdots \matsym{W}_2 (\matsym{W}_1 \vecsym{x} + \vecsym{b}_1) + \vecsym{b}_2) \cdots) + \vecsym{b}_{k-1}) + \vecsym{b}_k
    \end{equation}
    となる.
    かっこを展開して整理すると
    \begin{equation}
      f(\vecsym{x}) = \matsym{W}_k \matsym{W}_{k-1} \cdots \matsym{W}_2 \matsym{W}_1 \vecsym{x} + \matsym{W}_k \matsym{W}_{k-1} \cdots \matsym{W}_2 \vecsym{b}_1 + \matsym{W}_k \matsym{W}_{k-1} \cdots \matsym{W}_3 \vecsym{b}_2 + \cdots + \matsym{W}_k \vecsym{b}_{k-1} + \vecsym{b}_k
    \end{equation}
    が得られる.
    ここで,
    $\matsym{W}_k \matsym{W}_{k-1} \cdots \matsym{W}_2 \matsym{W}_1$
    および
    $\matsym{W}_k \matsym{W}_{k-1} \cdots \matsym{W}_2 \vecsym{b}_1 + \matsym{W}_k \matsym{W}_{k-1} \cdots \matsym{W}_3 \vecsym{b}_2 + \cdots + \matsym{W}_k \vecsym{b}_{k-1} + \vecsym{b}_k$
    をそれぞれ,新たに$\matsym{W}$,$\vecsym{b}$とおくと,$f(\vecsym{x}) = \matsym{W} \vecsym{x} + \vecsym{b}$
    となる.
    これは,$f_1(\vecsym{x}) = \vecsym{x}$である1層のニューラルネットワークである.
    すなわち,$f_n(\vecsym{x}) = \vecsym{x}$のとき,
    $k$層のニューラルネットワークと等価な$1$層のニューラルネットワークが存在する.

