\section{線形代数の基礎}

\subsection{連立方程式}
  \vspace{1mm}
  \begin{enumerate}[label=(\roman*)]
    \item
    拡大係数行列$\tilde{\mathrm{A}} = (\mathrm{A}|b)$に対し,拡大係数行列の左側が以下のような行列
    (対角に相当する成分のみ1を含み,他の成分がすべて0である行列) となるよう,
    行基本変形を施す.
    \begin{equation}
      \left(
        \begin{array}{rrrrrrr}
          1 &   &        &   &   &        & 0 \\
            & 1 &        &   &   &        &   \\
            &   & \ddots &   &   &        &   \\
            &   &        & 1 &   &        &   \\
            &   &        &   & 0 &        &   \\
            &   &        &   &   & \ddots &   \\
          0 &   &        &   &   &        & 0 \\
        \end{array}
      \right)
    \end{equation}
    この結果得られる拡大係数行列の右側が元の方程式の解に相当する.
    よって解$x$は,
    \begin{equation}
      x = \left(
        \begin{array}{r}
          6 \\ 2 \\ 3
        \end{array}
        \right)
    \end{equation}
    である.
    
    \item
    \begin{itemize}
      \item (i)と同様に拡大係数行列$\tilde{\mathrm{A}} = (\mathrm{A}|b)$を変形すると以下を得る
        (変形を途中でやめていることに注意).
        \begin{equation}
          \left(
            \begin{array}{rr|r}
              1 & 1 & 1 \\
              0 & 1 & 0 \\
              0 & 0 & 2 \\
            \end{array}
          \right)
        \end{equation}
        このことから,
        $\rank(\mathrm{A}) \neq \rank(\tilde{\mathrm{A}})$であり,
        $Ax - b = 0$は解を持たない.
      \item $\|\mathrm{A}x-b\|$ を最小とする $x$ は,
      $\|\mathrm{A}x-b\|^2$ を最小とする $x$ と等しい.
      よって,$\|\mathrm{A}x-b\|^2$ を最小とする $x$ を考える.
      $\|\mathrm{A}x-b\|^2$は二次式だから,
      \begin{equation}
        \frac{\partial}{\partial x} \| \mathrm{A}x-b \|^2 = 0
      \end{equation}
      を満たす$x$を求めればよい.
      上式を解くと,
      \begin{equation}
        x = (\mathrm{A}^\top \mathrm{A})^{-1} \mathrm{A}^\top b
      \end{equation}
      を得る.
      したがって,
      \begin{equation}
        x = \left(
          \begin{array}{r}
             1 \\
             \frac{2}{3} \\
          \end{array}
          \right).
      \end{equation}
      \begin{enumerate}[label=別解\arabic*]
        \item $\|\mathrm{A}x-b\|^2$を$x$の各要素について平方完成して最小値を求める
        \item $\|\mathrm{A}x-b\|^2$を$x$の各要素について微分して,
          \begin{equation}
            \frac{\partial}{\partial x_1} \| \mathrm{A}x-b \|^2
            = \frac{\partial}{\partial x_2} \| \mathrm{A}x-b \|^2 = 0
          \end{equation}
          を解き,最小値を求める.
      \end{enumerate}
    \end{itemize}

    \item
    \begin{itemize}
      \item (i)と同様に拡大係数行列$\tilde{\mathrm{A}} = (\mathrm{A}|b)$を変形すると以下を得る.
        \begin{equation}
          \left(
            \begin{array}{rrrr|r}
              1 & 0 & 0 &  2 & -1 \\
              0 & 1 & 0 & -1 &  1 \\
              0 & 0 & 1 & -1 & -1 \\
              0 & 0 & 0 &  0 &  0 \\
            \end{array}
          \right)
        \end{equation}
        よって,解$x$は,任意の$t \in \mathbb{R}$を用いて
        \begin{equation}
          x = t
            \left(
            \begin{array}{r}
              -2 \\ 1 \\ 1 \\ 1
            \end{array}
            \right)
            +
            \left(
            \begin{array}{r}
              -1 \\ 1 \\ -1 \\ 0
            \end{array}
            \right)
        \end{equation}
        として与えられる.
      \item $\|x\|$ を最小とする $x$ は,
      $\|x\|^2$ を最小とする $x$ と等しい.
      よって,$\|x\|^2$ を最小とする $x$ を考える.
      $\|x\|^2$は二次式だから,
      \begin{equation}
        \frac{d}{dt} \| x \|^2 = 0
      \end{equation}
      を満たす$x$を求めればよい.
      上式を解くと,
      \begin{equation}
        t = -\frac{2}{7}
      \end{equation}
      を得る.
      したがって,
      \begin{equation}
        x = -\frac{1}{7}
          \left(
          \begin{array}{r}
              3 \\
             -5 \\
              9 \\
              2
          \end{array}
          \right).
      \end{equation}
      \begin{enumerate}[label=別解\arabic*]
        \item $\|\mathrm{A}x-b\|^2$を$t$について平方完成して最小値を求める
      \end{enumerate}
    \end{itemize}
  \end{enumerate}

\subsection{線形変換}
  \vspace{1mm}
  \begin{enumerate}[label=(\roman*)]
    \item $\vecsym{y} = a \vecsym{x} + \vecsym{x}_\perp$より,
      $\vecsym{x}^\top \vecsym{y} = \vecsym{x}^\top (a \vecsym{x} + \vecsym{x}_\perp) = a \| \vecsym{x} \|^2$.
      $a$について整理すると,
      \begin{equation}
        a = \frac{1}{\| \vecsym{x} \|^2} \vecsym{x}^\top \vecsym{y}
      \end{equation}
      を得る.
    \item $\matsym{A} \vecsym{y} = a \vecsym{x}$に(i)の結果を代入することで,
      \begin{equation}
        \matsym{A} = \frac{1}{\| \vecsym{x} \|^2} \vecsym{x} \vecsym{x}^\top
      \end{equation}
      を得る.
  \end{enumerate}

\subsection{行列式と逆行列の存在条件}
  \vspace{1mm}
  \begin{enumerate}[label=(\roman*)]
    \item
      \begin{align}
        \det{(\matsym{V})} &=
          \begin{vmatrix}
            1 & 1 & 1 & 1 \\
            x_1 & x_2 & x_3 & x_4 \\
            x_1^2 & x_2^2 & x_3^2 & x_4^2 \\
            x_1^3 & x_2^3 & x_3^3 & x_4^3
          \end{vmatrix}
          =
          \begin{vmatrix}
            1 & 1 & 1 & 1 \\
            0 & x_2 - x_1 & x_3 - x_1 & x_4 - x_1 \\
            0 & x_2 (x_2 - x_1) & x_3 (x_3 - x_1) & x_4 (x_4 - x_1)\\
            0 & x_2^2 (x_2 - x_1) & x_3^2 (x_3 - x_1) & x_4^2 (x_4 - x_1) 
          \end{vmatrix}\\
          &=
          \begin{vmatrix}
            x_2 - x_1 & x_3 - x_1 & x_4 - x_1 \\
            x_2 (x_2 - x_1) & x_3 (x_3 - x_1) & x_4 (x_4 - x_1)\\
            x_2^2 (x_2 - x_1) & x_3^2 (x_3 - x_1) & x_4^2 (x_4 - x_1) 
          \end{vmatrix}\\
          &= (x_2 - x_1) (x_3 - x_1) (x_4 - x_1)
          \begin{vmatrix}
            1 & 1 & 1 \\
            x_2 & x_3 & x_4\\
            x_2^2 & x_3^2 & x_4^2
          \end{vmatrix}\\
          &= (x_2 - x_1) (x_3 - x_1) (x_4 - x_1)
          \begin{vmatrix}
            1 & 1 & 1 \\
            0 & x_3 - x_2 & x_4 - x_2\\
            0 & x_3 (x_3 - x_2) & x_4 (x_3 - x_2)
          \end{vmatrix}\\
          &= (x_2 - x_1) (x_3 - x_1) (x_4 - x_1) (x_3 - x_2) (x_4 - x_2)
          \begin{vmatrix}
            1 & 1 \\
            x_3 & x_4
          \end{vmatrix}\\
          &= (x_2 - x_1) (x_3 - x_1) (x_4 - x_1) (x_3 - x_2) (x_4 - x_2) (x_4 - x_3)
      \end{align}
    \item 行列$\matsym{V}$が逆行列を持つための必要十分条件は,$\det{(\matsym{V})} \neq 0$であることである.
      \begin{equation}
        \det{(\matsym{V})} = (x_2 - x_1) (x_3 - x_1) (x_4 - x_1) (x_3 - x_2) (x_4 - x_2) (x_4 - x_3) \neq 0
      \end{equation}
      より,
      $i, j \in \{1, 2, 3, 4\}$かつ$i \neq j$であるすべての$(i, j)$について$x_i \neq x_j$を満たすとき,
      行列$\matsym{V}$は逆行列を持つ.
  \end{enumerate}
      
\clearpage
\subsection{回転行列}
  $\vecsym{e}_1 = \begin{pmatrix} 1 \\ 0 \end{pmatrix}$,
  $\vecsym{e}_2 = \begin{pmatrix} 0 \\ 1 \end{pmatrix}$
  とする.
  \begin{enumerate}[label=(\roman*)]
    \item $\vecsym{e}'_1 = \begin{pmatrix} \cos\theta \\ \sin\theta \end{pmatrix}$,
      $\vecsym{e}'_2 = \begin{pmatrix} -\sin\theta \\ \cos\theta \end{pmatrix}$
    \item 
      \begin{equation}
        \begin{pmatrix} \vecsym{e}'_1 & \vecsym{e}'_2 \end{pmatrix}
        = \matsym{R}_{\theta} \begin{pmatrix} \vecsym{e}_1 & \vecsym{e}_2 \end{pmatrix}
      \end{equation}
      より,
      \begin{equation}
        \begin{pmatrix}
          \cos\theta & -\sin\theta \\
          \sin\theta & \cos\theta
        \end{pmatrix}
        = \matsym{R}_{\theta}
        \begin{pmatrix}
          1 & 0 \\
          0 & 1
        \end{pmatrix}.
      \end{equation}
      したがって,
      \begin{equation}
        \matsym{R}_\theta =
          \begin{pmatrix}
            \cos\theta & -\sin\theta \\
            \sin\theta & \cos\theta
          \end{pmatrix}
      \end{equation}
    \item
      \begin{align}
        \vecsym{\nu} &= \matsym{R}_\phi \matsym{R}_\theta \vecsym{e}_1 \\
          &=
          \begin{pmatrix}
            \cos\theta & -\sin\theta \\
            \sin\theta & \cos\theta
          \end{pmatrix}
          \begin{pmatrix}
            \cos\theta & -\sin\theta \\
            \sin\theta & \cos\theta
          \end{pmatrix}
          \vecsym{e}_1\\
          &=
          \begin{pmatrix}
            \cos\theta \cos\phi - \sin\theta \sin\phi \\
            \sin\theta \cos\phi + \cos\theta \sin\phi
          \end{pmatrix}
      \end{align}
    \item
      \begin{equation}
        \matsym{R}_{-\theta} 
          =
          \begin{pmatrix}
            \cos\theta  & \sin\theta \\
            -\sin\theta & \cos\theta
          \end{pmatrix}
      \end{equation}
      であり,
      \begin{equation}
        \matsym{R}_\theta \matsym{R}_{-\theta} = \matsym{R}_{\theta} \matsym{R}_\theta = \matsym{I}
      \end{equation}
      ($\matsym{I}$は単位行列)
      より,$\matsym{R}_\theta^{-1} = \matsym{R}_{-\theta}$.
  \end{enumerate}