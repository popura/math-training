\section{線形時不変システム}

\subsection{線形性}

\subsection{時不変性}

\subsection{線形時不変システム}

\subsection{線形時不変システムの差分方程式表現}

\subsection{インパルス応答}

\subsection{周波数応答}

\subsection{因果性}

\subsection{FIRフィルタ}

\subsection{IIRフィルタ}

\subsection{フィルタの周波数応答}
  \begin{enumerate}[label=(\roman*)]
    \item フィルタのインパルス応答が$h[0] = 1, h[1] = -0.97$であることより,
      周波数応答$H(\omega)$(インパルス応答の離散時間フーリエ変換)は
      \begin{align}
        H(\omega) &= \sum_{n = -\infty}^{\infty} h[n] e^{-j\omega n} \\
                  &= 1 - 0.97 e^{-j \omega}
      \end{align}
      となる.ここで,$j$は虚数単位,$\omega$は正規化角周波数をそれぞれ表す.
      このフィルタの振幅特性$|H(\omega)|$は,
      \begin{align}
        |H(\omega)| &= | 1 - 0.97 e^{-j \omega}| \\
                    &= | 1 - 0.97 (\cos \omega - j \sin \omega)| \\
                    &= \sqrt{(1 - 0.97 \cos \omega)^2 + (0.97 j \sin \omega)^2} \\
                    &= \sqrt{1 - 2 \times 0.97 \cos \omega + 0.97^2 \cos^2 \omega
                     + 0.97^2 \sin^2 \omega} \\
                    &= \sqrt{(1 + 0.97^2) - 2 \times 0.97 \cos \omega}
      \end{align}
      である.
      $H(\omega)$は周期$2\pi$を持ち,$|H(\omega)|$は偶対象である.
      よって,$0 \le \omega < \pi$の範囲のみを考える.
      $|H(\omega)|$は$0 \le \omega \pi$の範囲で単調増加し,
      $\omega = 97 / 200 < \pi$のとき$|H(\omega) = 1|$となる.
      したがって,このフィルタ$h$は,
      $\omega > 97 / 200$の周波数成分を強調し
      $\omega < 97 / 200$の周波数成分を減衰させる高域強調フィルタである.
  \end{enumerate}

\subsection{IIRフィルタの安定性}
