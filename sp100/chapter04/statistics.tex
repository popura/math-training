\section{確率統計の基礎}

\subsection{確率モーメント}
  確率変数 $X$ が,それぞれ1/6 の確率で,$-3, -2, -1, 1, 2, 3$ のいずれかの値をとるものとします.
  \begin{enumerate}[label=(\roman*)]
    \item 期待値 $E[X], E[X^2]$をそれぞれ求めてください.
    \item $E[aX+b] = aE[X] + b$ であることを示してください.
    \item 確率変数 $X$ の分散は $\mathrm{Var}(X) = E[X^2] - E[X]^2$  であることを証明してください.
  \end{enumerate}

\subsection{確率分布}

\subsection{分散共分散行列}
  \begin{enumerate}[label=(\roman*)]
    \item エルミート行列
    \item 半正定値
  \end{enumerate}

\subsection{最小二乗法の閉形式解}
  $\matsym{A}$を実行列,$\vecsym{b}$を実列ベクトルとし,
  \begin{equation}
    \mathcal{L}(\vecsym{x}) = \| \matsym{A}\vecsym{x} - \vecsym{b} \|^2
    \label{eq:least_square}
  \end{equation}
  を最小にする列ベクトル$\vecsym{x}$を
  求めることを考える(最小二乗法).
  ただし,$\matsym{A}, \vecsym{b}, \vecsym{x}$は,式(\ref{eq:least_square})の計算ができるサイズを持つものとする.
  このとき,以下の問いに答えよ.
  \begin{enumerate}[label=(\roman*)]
    \item $\mathcal{L}(\vecsym{x})$を$\vecsym{x}$について微分せよ
      (すなわち,$\frac{\partial}{\partial \vecsym{x}} \mathcal{L}(\vecsym{x})$を計算せよ).
    \item (i)の結果から,
      $\mathcal{L}(\vecsym{x})$を最小にする$\vecsym{x}$が満たすべき条件を答えよ.
    \item $\tp{\matsym{A}}\matsym{A}$が正則であるとき,
      $\vecsym{x}$を求めよ.
  \end{enumerate}

\subsection{最尤推定}

\subsection{主成分分析}
