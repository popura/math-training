\section{微積分}

\subsection{勾配ベクトル}
    関数 $f: \mathbb{R}^n \rightarrow \mathbb{R}$ のベクトル $x \in \mathbb{R}^n$ に関する微分(勾配)を,
    \begin{equation}
      \frac{\partial f}{\partial x} = \left(
        \frac{\partial f}{\partial x_1} \frac{\partial f}{\partial x_2} \cdots \frac{\partial f}{\partial x_n}
        \right)^\top
    \end{equation}
    と定義します.
    また,関数 $g: \mathbb{R}^n \rightarrow \mathbb{R}^m$ のベクトル $x \in \mathbb{R}^n$ に関する微分(勾配)を,
    \begin{equation}
      \frac{\partial g}{\partial x} = \left(
        \begin{array}{rrrr}
          \frac{\partial g_1}{\partial x_1} & \frac{\partial g_2}{\partial x_1} & \cdots & \frac{\partial g_m}{\partial x_1} \\
          \frac{\partial g_1}{\partial x_2} & \ddots & & \vdots \\
          \vdots & & \ddots & \vdots \\
          \frac{\partial g_1}{\partial x_n} & \cdots & \cdots & \frac{\partial g_m}{\partial x_n}
        \end{array}
        \right)^\top
    \end{equation}
    と定義します.
    ベクトル $$a, x \in \mathbb{R}^n$$ を縦ベクトルとし,
    行列 $\mathrm{A}, \mathrm{E} \in \mathbb{R}^{n \times n}$ をそれぞれ対称行列,単位行列とするとき,
    以下の等式を証明してください.
    \begin{enumerate}[label=(\roman*)]
      \item $\frac{\partial}{\partial x} x^\top = \mathrm{E}$
      \item $\frac{\partial}{\partial x} x^\top a = a$
      \item $\frac{\partial}{\partial x} a^\top x = a$
      \item $\frac{\partial}{\partial x} x^\top \mathrm{A} x = 2 \mathrm{A} x$
    \end{enumerate}

\subsection{勾配,行列}
  実数のスカラー関数 $g(\mathbf X)$ の行列 $\mathbf X$ についての偏微分が
  \begin{equation}
    \frac{\partial g}{\partial \mathbf X} =
    \begin{bmatrix}
      \displaystyle
      \frac{\partial g}{\partial x_{11}}&\cdots&\displaystyle \frac{\partial g}{\partial x_{1N}}\\
      \vdots& &\vdots\\
      \displaystyle \frac{\partial g}{\partial x_{M1}}&\cdots&\displaystyle \frac{\partial g}{\partial x_{MN}}
    \end{bmatrix}
  \end{equation}
  で定義されるとき,以下の等式を証明せよ.
  \begin{enumerate}[label=(\roman*)]
    \item $\frac{\partial}{\partial \mathbf{X}}{\mathrm{tr}}(\mathbf{X})={\mathbf{I}}$
    \item $\frac{\partial}{\partial \mathbf{X}}{\mathrm{tr}}({\mathbf{A}}{\mathbf{X}})={\mathbf{A}}^{\top}$
    \item $\frac{\partial}{\partial \mathrm{X}}{\mathrm{tr}}({\mathbf{A}}{\mathbf{X}}^{\top})={\mathbf{A}}$
    \item $\frac{\partial}{\partial \mathbf{X}}{\mathrm{tr}}({\mathbf{A}}{\mathbf{X}}{\mathbf{B}})={\mathbf{A}}^{\top}{\mathbf{B}}^{\top}$
  \end{enumerate}
  ただし,$\mathbf{A}$は各行列積が定義できるサイズとする.

\subsection{最急降下法}
  $x \in \numset{R}$とし,関数$f(x) = \frac{1}{6}x^6 - \frac{3}{5}x^5 - x^4 + 4x^3$を,
  最急降下法を用いて最小化することを考える.
  最急降下法を用いたパラメータの更新式は,$x_n = x_{n-1} - \alpha f'(x_{n-1})$である.
  以下の問いに答えよ.
  ただし,解答にはプログラムを用いてよい(プログラムも提出すること).
  \begin{enumerate}[label=(\roman*)]
    \item $x_0 = \frac{1}{2}, \alpha = 0.2$として,$f(x)$を最小化せよ.
    \item $x_0 = 4, \alpha = 0.002$として,$f(x)$を最小化せよ.
    \item $x_0 = -3, \alpha = 0.01$として,$f(x)$を最小化せよ.
    \item $x_0 = -3, \alpha = 0.02$として,$f(x)$を最小化せよ.
    \item $f(x)$の増減を手計算で調べ,上記の結果を考察せよ.
  \end{enumerate}

\subsection{微分の連鎖律}
  $f = f_2(\mathrm{W}_2 f_1(\mathrm{W}_1 x + b_1) + b_2)$ とします.
  \begin{enumerate}[label=(\roman*)]
    \item $\frac{\partial f}{\partial A_2}, \frac{\partial f}{\partial b_2}$ を計算してください.
    \item $\frac{\partial f}{\partial A_1}, \frac{\partial f}{\partial b_1}$ を計算してください.
    \item $\frac{\partial f}{\partial x}$ を計算してください
  \end{enumerate}
