\section{線形代数の基礎}

\begin{enumerate}[label=問\arabic*.]
  \vspace{2mm}
  %
  \item 連立方程式 \label{linalg:linear_system}

  \vspace{1mm}
  \begin{enumerate}[label=(\roman*)]
    \item
    \begin{equation}
      \matsym{A} =
        \begin{pmatrix}
          -2 & -4 & -3 \\
           3 &  3 & -4 \\
          -5 &  2 &  3
        \end{pmatrix},
      \vecsym{b} =
        \begin{pmatrix}
          -29 \\
           12 \\
          -17
        \end{pmatrix},
      \vecsym{x} \in \numset{R}^3
    \end{equation}
    とする.
    このとき,方程式 $\matsym{A} \vecsym{x} - \vecsym{b} = \vecsym{0}$ を解け.
    
    \item
    \begin{equation}
      \matsym{A} =
        \begin{pmatrix}
           1 & 1 \\ 1 & -1 \\ 0 & 1
        \end{pmatrix},
      \vecsym{b} =
        \begin{pmatrix}
           1 \\
           1 \\
           2
        \end{pmatrix},
      \vecsym{x} \in \numset{R}^2
    \end{equation}
    とする.
    \begin{itemize}
      \item このとき,方程式 $\matsym{A} \vecsym{x} - \vecsym{b} = \vecsym{0}$ を解け.
      \item $n$ 次元ベクトル $\vecsym{v} = \tp{(v_1, \cdots, v_n)}$ のノルム $\| \vecsym{v} \|$ を
        $\| \vecsym{v} \| = \sqrt{\sum_{i=1}^n v_i^2}$ とする.
        $\|\matsym{A} \vecsym{x} - \vecsym{b} \|$ が最小となる $\vecsym{x}$ を求めよ.
    \end{itemize}
    
    \item
    \begin{equation}
      \matsym{A} =
        \begin{pmatrix}
          2 & 4 & 1 & -1 \\
          1 & 2 & -1 & 1 \\
          2 & 1 & 1 & 2 \\
          1 & 3 & 2 & -3
        \end{pmatrix},
      \vecsym{b} =
        \begin{pmatrix}
          1 \\ 2 \\ -2 \\ 0
        \end{pmatrix},
      \vecsym{x} \in \numset{R}^4
    \end{equation}
    とする.
    \begin{itemize}
      \item このとき,方程式 $\matsym{A} \vecsym{x} - \vecsym{b} = \vecsym{0}$ を解け.
      \item $n$ 次元ベクトル $\vecsym{v} = \tp{(v_1, \cdots, v_n)}$ のノルム $\|\vecsym{v}\|$ を
        $\|\vecsym{v}\| = \sqrt{\sum_{i=1}^n v_i^2}$ とする.
        $\matsym{A} \vecsym{x} - \vecsym{b} = \vecsym{0}$の解$\vecsym{x}$のうち,$\|\vecsym{x}\|$が最小となるものを求めよ.
    \end{itemize}
  \end{enumerate}
  %
  \item 線形変換

  \vspace{1mm}
  $\vecsym{x}$,$\vecsym{y}$を互いに線形独立な縦ベクトルとする.
  いま$\vecsym{y}$を,
  \begin{equation}
    \vecsym{y} = a \vecsym{x} + \vecsym{x}_\perp
  \end{equation}
  のように表したい.
  ここで,$a$はスカラー,$\vecsym{x}_{\perp}$は$\vecsym{x}$と直交するベクトルを表す.
  すなわち,$\tp{\vecsym{x}} \vecsym{x}_{\perp} = 0$である.
  \begin{enumerate}[label=(\roman*)]
    \item $a$を求めよ.
    \item $\vecsym{y}$から$a \vecsym{x}$を得る演算は,ある行列$\matsym{A}$を用いて,
      $\matsym{A}\vecsym{y} = a \vecsym{x}$とかくことができる.$\matsym{A}$を求めよ.
  \end{enumerate}
  %
  \item 行列のランク

  \vspace{1mm}
  2つの行列$\matsym{A}$,$\matsym{B}$の積として,
  $\matsym{C}=\matsym{A}\matsym{B}$で表される行列$\matsym{C}$のランクについて考える.
  ただし,行列$\matsym{C}$のサイズは$5\times 4$である.

  \begin{enumerate}[label=(\roman*)]
    \item 行列$\matsym{A}$,$\matsym{B}$として,
      それぞれ$5\times 1$,$1\times 4$の適当な行列を自分で考え,$\matsym{C}$を計算しなさい.
      ただし,$\matsym{A}$,$\matsym{B}$は零行列(すべての要素が0の行列)ではないものとします.
    \item (i)のとき、行列$\matsym{C}$のランクを求めなさい.
      理由をつけて答えても,(i)の答えを階段化するなどして答えても結構です.
    \item 行列$\matsym{A}$,$\matsym{B}$として,
      それぞれ$5\times 2$,$2\times 4$の適当な行列を自分で考え,$\matsym{C}$を計算しなさい.
      ただし,$\matsym{A}$,$\matsym{B}$はフルランクの行列($m\times n$行列のランクが$m$と$n$の小さいほうに等しいとき,この行列をフルランクであるという)であるものとします.
    \item (iii)のとき、行列$\matsym{C}$のランクを求めなさい.
      理由をつけて答えても,(iii)の答えを階段化するなどして答えても結構です.
    \item 行列$\matsym{A}$,$\matsym{B}$が,
      それぞれ$5\times N$,$N\times 4$のサイズであったとき,
      行列$\matsym{C}$のランクがいくつになるか答えなさい.証明なしでも結構です.
      ただし,$N\in \numset{N}$で,$\matsym{A}$,$\matsym{B}$はフルランクの行列であるとします.
      必要であれば、$N$で場合分けしてください.
  \end{enumerate}
  %
  \item 逆行列,固有値分解

  \vspace{1mm}
  ベクトル$\vecsym{x} \in \numset{R}^2$を原点中心に$\theta$だけ(反時計回りに)回転させたベクトル$\vecsym{y}$は,
  行列$\matsym{R}_\theta$を用いて$\vecsym{y} = \matsym{R}_\theta \vecsym{x}$と与えられる.
  \begin{enumerate}[label=(\roman*)]
    \item $\matsym{R}_\theta$を求めよ.
    \item $\matsym{R}_\theta$の逆行列を求めよ.
    \item $\matsym{R}_\theta$の固有値,固有ベクトルを求めよ.
    \item 正則行列$\matsym{U}$および対角行列$\matsym{\Lambda}$を用いて,
      $\matsym{R}_\theta = \matsym{U}\matsym{\Lambda} \matsym{U}^{-1}$の形で$\matsym{R}_\theta$を表せ.
  \end{enumerate}

  \item エルミート行列の性質

  \vspace{1mm}
  エルミート行列$\matsym{A}$の固有値分解$\matsym{A} = \matsym{U}\matsym{\Lambda}\matsym{U}^{-1}$を考える.
  \begin{enumerate}[label=(\roman*)]
    % \item エルミート行列$\matsym{A}$の異なる固有値に対応する固有ベクトルが,任意の内積について直交することを示せ.
    \item 任意の複素ベクトル$\vecsym{v}=\tp{(v_1, \cdots, v_n)}$のとき,
      $\htp{\vecsym{v}} \vecsym{v} \geq 0$を証明せよ.
      ここで, $\htp{\cdot}$ はエルミート転置を表す.
    \item もしくはエルミート行列の固有値$\lambda$が実数になることを証明せよ.
    \item エルミート行列$\matsym{A}$の固有値分解が
      $\matsym{A} = \matsym{U}\matsym{\Lambda}\htp{\matsym{U}}$ として与えられることを示せ.
    \item フルランクを持つエルミート行列$\matsym{A}$の逆行列$\matsym{A}^{-1}$が,
      $\matsym{A}^{-1} = \matsym{U}\matsym{\Lambda}^{-1}\htp{\matsym{U}}$として与えられることを示せ.
  \end{enumerate}

  \item 特異値分解 \label{linalg:svd}

  \vspace{1mm}
  行列$\matsym{A}$は,各行が直交する行列$\matsym{U}, \matsym{V}$,
  および対角行列$\matsym{\Sigma}$を用いて
  $\matsym{A}=\matsym{U}\matsym{\Sigma}\htp{\matsym{V}}$の形に分解できる.
  これを特異値分解と呼ぶ.ここで,$\htp{\cdot}$はエルミート転置を表す.
  \begin{enumerate}[label=(\roman*)]
    \item
      \begin{equation}
        \matsym{A} =
          \begin{pmatrix}
            1 & 1 \\
            1 & -1 \\
            0 & 1
          \end{pmatrix}
      \end{equation}
      の特異値分解を求めよ.
    \item
      \begin{equation}
        \matsym{A} =
          \begin{pmatrix}
            2 & 4 & 1 &-1 \\
            1 & 2 & -1 & 1 \\
            2 & 1 & 1 & 2 \\
            1 & 3 & 2 & -3
          \end{pmatrix}
      \end{equation}の特異値分解を求めよ.
      ただし,この特異値分解に限り,プログラム等を用いて解答してもよい.
  \end{enumerate}

  \item Moore-Penroseの擬似逆行列

  \vspace{1mm}
  行列$\matsym{A}$について,以下の4条件
  \begin{itemize}
    \item $\matsym{A}\pinv{\matsym{A}}\matsym{A}=\matsym{A},$
    \item $\pinv{\matsym{A}}\matsym{A}\pinv{\matsym{A}}=\pinv{\matsym{A}},$
    \item $\htp{(\matsym{A}\pinv{\matsym{A}})}=\matsym{A}\pinv{\matsym{A}},$
    \item $\htp{(\pinv{\matsym{A}}\matsym{A})}=\pinv{\matsym{A}}\matsym{A}$
  \end{itemize}
  を満たす行列$\pinv{\matsym{A}}$をMoore-Penroseの擬似逆行列と呼ぶ.
  ここで,$\matsym{A}$が正則行列ならば,$\pinv{\matsym{A}}=\matsym{A}^{-1}$を満たす.
  また,$\matsym{A}$の特異値分解が
  $\matsym{A}=\matsym{U}\matsym{\Sigma}\htp{\matsym{V}}$として与えられるとき,
  $\pinv{\matsym{A}}=\matsym{V}\pinv{\matsym{\Sigma}}\htp{\matsym{U}}$が成り立つ.
  \ref{linalg:svd}の結果を用いて以下の問いに答えよ.
  \begin{enumerate}[label=(\roman*)]
    \item
      \begin{equation}
        \matsym{A} =
          \begin{pmatrix}
            1 & 1 \\
            1 & -1 \\
            0 & 1
          \end{pmatrix}
      \end{equation}
      について,$\pinv{\matsym{A}}$を求めよ.
      また,$\vecsym{b} = \begin{pmatrix} 1 \\ 1 \\ 2 \end{pmatrix}$のとき,
      $\pinv{\matsym{A}} \vecsym{b}$を計算し,
      その結果を\ref{linalg:linear_system}(ii)で計算した$\vecsym{x}$と比較せよ.
  
    \item
      \begin{equation}
        \matsym{A} =
          \begin{pmatrix}
            2 & 4 & 1 & -1 \\
            1 & 2 & -1 & 1 \\
            2 & 1 & 1 & 2 \\
            1 & 3 & 2 & -3
          \end{pmatrix}
      \end{equation}
      について,$\pinv{\matsym{A}}$を求めよ.
      また,$\vecsym{b}=\begin{pmatrix} 1 \\ 2 \\ -2 \\ 0 \end{pmatrix}$のとき,
      $\pinv{\matsym{A}} b$を計算し,その結果を\ref{linalg:linear_system}(iii)で計算した$\vecsym{x}$と比較せよ.
      ただし,この計算に限り,プログラム等を利用して解答してもよい.
  \end{enumerate}

  \item 逆行列

    \begin{equation}
      \matsym{H} =
        \begin{pmatrix}
                     1 & \frac{1}{2} & \frac{1}{3} \\
           \frac{1}{2} & \frac{1}{3} & \frac{1}{4} \\
           \frac{1}{3} & \frac{1}{4} & \frac{1}{5}
        \end{pmatrix}
    \end{equation}
    とする.
    \begin{enumerate}{label=(\roman*)}
      \item $\matsym{H}$の逆行列$\matsym{H}^{-1}$をコンピュータで計算せよ.
      \item $\matsym{H}$の逆行列$\matsym{H}^{-1}$を手計算せよ.
    \end{enumerate}

  \item 行列式と逆行列の存在条件

    $i, j \in \mathbb{N}, a_{ij} \in \mathbb{C}$ のとき,
    行列 $\mathrm{A}= \begin{pmatrix} a_{11} & a_{12} & \cdots & a_{1n} \\ a_{21} & a_{22} & \cdots & a_{2n} \\ \vdots & \vdots & & \vdots \\ a_{n1} & a_{n2} & \cdots & a_{nn} \end{pmatrix} $
    の行列式を$\det{(\mathrm{A})} = \begin{vmatrix} a_{11} & a_{12} & \cdots & a_{1n} \\ a_{21} & a_{22} & \cdots & a_{2n} \\ \vdots & \vdots & & \vdots \\ a_{n1} & a_{n2} & \cdots & a_{nn} \end{vmatrix}$ と表します.
    - $$i \in \mathbb{N}, x_i \in \mathbb{C}, \mathrm{V}= \left( \begin{array}{rrrr} 1 & 1 & 1 & 1 \\ x_1 & x_2 & x_3 & x_4 \\ x_1^2 & x_2^2 & x_3^2 & x_4^2 \\ x_1^3 & x_2^3 & x_3^3 & x_4^3 \end{array} \right)$$ のとき,
        $$\det{(\mathrm{V})}$$を計算してください.
    - $$\mathrm{V}$$の逆行列が存在するための条件を示してください.

  \item 対角行列,三角行列の行列式

    対角行列,上三角行列,下三角行列の行列式を計算してください.

  \item 固有値,固有ベクトル
    4次元ベクトル $$x = \left( \begin{array}{r} a \\ b \\ c \\ d \end{array} \right)$$ に対し,要素を巡回的にシフトした $$y = \left( \begin{array}{r} b \\ c \\ d \\ a \end{array} \right)$$を考えます.
    - $$y = \mathrm{P}x$$ となるような,4x4の行列 $$\mathrm{P}$$ を求めてください.
    - $$\mathrm{P}v = \lambda v$$ ($$\lambda$$はスカラー) となるようなベクトル $$v$$,すなわち,$$\mathrm{P}$$ の固有ベクトルをすべて求めてください.
  
\end{enumerate}